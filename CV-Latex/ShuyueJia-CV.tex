\documentclass{my_cv}

\begin{document}

\hspace*{\fill}

\name{Shuyue Jia} 

\hspace*{\fill}

\longcontact
{\href{mailto:shuyuej@ieee.org}{\nolinkurl{shuyuej@ieee.org}}}
{(+852) 54604494}
{\href{https://github.com/SuperBruceJia}{GitHub}}
{\href{https://resume.github.io/?SuperBruceJia}{GitHub Resume}}
{\href{http://shuyuej.com/blog}{Blog}}
{\href{https://shuyuej.com/}{Personal Profile}}

\hspace*{\fill} 

\section{Education}
\datedsubsection{City University of Hong Kong (CityU), Hong Kong}{}{May 2021 - Present}
\workitemstwo
{Master of Philosophy - M.Phil. Computer Science}
{Research area: Image Quality Assessment \& Perceptual Optimization}

\datedsubsection{Northeast Electric Power University (NEEPU), Jilin, China}{}{Sep 2016 - Jun 2020}
\workitemstwo
{Bachelor of Engineering, Intelligence Science and Technology Major, GPA: 80.26/100}
{Supervisor: \href{mailto:ymh7821@163.com}{Prof. Yimin Hou}, Research area: EEG Signals Classification based on Deep Learning Methods}

\datedsubsection{University of California, Irvine (UC Irvine), CA, USA}{}{Jul - Sep 2017}
\workitemstwo
{Summer School, Computer Science, GPA: 4.0/4.0}
{Selected coursework: Computer Systems and Architecture (A+), University Writing and Communication (Pass)}

\hspace*{\fill} 

\section{Experience}
\datedsubsection{Tencent Video, Beijing}{Recommendation System Intern}{Oct - Dec 2020}
\workitemstwo
{Assist with the unified architecture w.r.t. the \emph{Rerank} Module for Tencent Video Recommendation System (RS).}
{Conducted research on the Dynamic Graph Convolutional Neural Networks (DGCN) \href{https://shuyuej.com/files/Dynamic-GCN-Survey.pdf}{Survey} and learned Reinforcement Learning models.}

\datedsubsection{Philips Research, Shanghai}{Natural Language Processing Intern}{Jul - Oct 2020}
\workitemsthree
{\href{https://github.com/SuperBruceJia/Medical-Concept-Mapping}{Medical Concept Mapping}: three levels $\rightarrow$ BPE and FMM \& BMM Algorithms for Sub-words (Syntax-level), Word Vector Cosine Similarity (Semantics-level), and Knowledge Graph (Pragmatics-level).}
{\href{https://github.com/SuperBruceJia/MedicalNER}{Medical NER}: compared the performances of different models $\rightarrow$ CRF++, Character-level BiLSTM + CRF, Character-level BiLSTM + Word-level BiLSTM / CNNs + CRF, and deployed the models using Flask and Docker as web apps. Codes are available \href{https://github.com/SuperBruceJia/pytorch-flask-deploy-webapp}{here}, and the Docker Images are available on \href{https://hub.docker.com/u/shuyuej}{Docker Hub}.}
{\href{https://github.com/SuperBruceJia/dynamic-web-crawlering-python}{Dynamic Webs Crawling}: learned and crawled 620,000 words from NSTL using Python parallel package threading and other tricks to prevent Anti-reptile.}

\datedsubsection{Tsinghua University, Beijing}{Summer Intern}{Jun - Aug 2019}
\workitemstwo
{I was in a team that was responsible for building a salesman training system, which was a piece of insurance dialogue systems. During intern, I led the effort to create a \href{https://github.com/SuperBruceJia/Chinese-Chat-Title-NER-BERT-BiLSTM-CRF}{Chinese Chat Title Named Entity Recognition (NER)} via the BERT-BiLSTM-CRF model, and then matched the formal name with the recognized title through rules. (NER Dataset: 30,676 samples, 96.73\% accuracy on 550 samples.)}
{I also assisted in testing the sales training review system, and integrated salesman’s dialogue according to different difficulty levels, in verifying the reliability of the system.}

\hspace*{\fill} 

\section{Research}
\workitemsone
{A Novel Approach of Decoding EEG Four-Class Motor Imagery Tasks via Scout ESI and CNN. \href{https://iopscience.iop.org/article/10.1088/1741-2552/ab4af6/meta}{[Paper]} \href{https://github.com/SuperBruceJia/EEG-Motor-Imagery-Classification-CNNs-TensorFlow}{[Code]}\\
	Yimin Hou, Lu Zhou, \textbf{Shuyue Jia}, and Xiangmin Lun. \\
	\emph{Journal of Neural Engineering}, 2020; 17(1):016048.}

\hspace*{\fill} 

\workitemsone
{GCNs-Net: A Graph Convolutional Neural Network Approach for Decoding Time-resolved EEG Motor Imagery Signals. \href{https://arxiv.org/abs/2006.08924}{[Paper]} \href{https://shuyuej.com/files/GCNs-Net.pdf}{[Spectral-GCN-Presentation]} \href{https://shuyuej.com/files/Dynamic-GCN-Survey.pdf}{[Dynamic-GCN-Presentation]} \href{https://github.com/SuperBruceJia/EEG-DL}{[Code]}\\
	Xiangmin Lun, \textbf{Shuyue Jia (Corresponding Author)}, Yimin Hou, Yan Shi, and Yang Li.\\
	\emph{arXiv preprint arXiv:2006.08924}, 2021. (Under Review)}

\hspace*{\fill} 

\workitemsone
{Deep Feature Mining via Attention-based BiLSTM-GCN for Human Motor Imagery Recognition. \href{https://arxiv.org/abs/2005.00777}{[Paper]}\href{https://github.com/SuperBruceJia/EEG-DL}{[Code]}\\
	Yimin Hou, \textbf{Shuyue Jia (Corresponding Author)}, Xiangmin Lun, Yan Shi, and Yang Li.\\
	\emph{arXiv preprint arXiv:2005.00777}, 2021. (Under Review)}

\hspace*{\fill}

\workitemsone
{Attention-based Graph ResNet for Motor Intent Detection from Raw EEG signals. \href{https://arxiv.org/abs/2007.13484}{[Paper]}\href{https://github.com/SuperBruceJia/EEG-DL}{[Code]}\\
	\textbf{Shuyue Jia (Corresponding Author)}, Yimin Hou, Yan Shi, and Yang Li.\\
	\emph{arXiv preprint arXiv:2007.13484}, 2021. (Rejected by MICCAI 2020)}

\hspace*{\fill}

\workitemsone
{Improving Performance: a Collaborative Strategy for the Multi-data Fusion of Electronic Nose and Hyperspectral to Track the Quality Difference of Rice. \href{https://www.sciencedirect.com/science/article/abs/pii/S0925400521001143}{[Paper]} \\
	Yan Shi, Hangcheng Yuan, Chenao Xiong, \textbf{Shuyue Jia}, Jingjing Liu, and Hong Men.\\
	\emph{Sensors \& Actuators: B. Chemical}, 2021; 129546.}

\hspace*{\fill}

\workitemsone
{Origin Traceability of Rice based on an Electronic Nose Coupled with a Feature Reduction Strategy. \href{https://iopscience.iop.org/article/10.1088/1361-6501/abb9e7/meta}{[Paper]} \\
	Yan Shi, Xiaofei Jia, Hangcheng Yuan, \textbf{Shuyue Jia}, Jingjing Liu, and Hong Men.\\
	\emph{Measurement Science and Technology}, 2020; 32(2):025107.}

\hspace*{\fill}

\section{Selected Projects}
\noindent \textbf{EEG-DL: A Deep Learning library for EEG Tasks (Signals) Classification} \href{https://github.com/SuperBruceJia/EEG-DL}{[Code]} \hfill May 2020 
\workitemsfour
{EEG-DL is a Deep Learning (DL) library written by TensorFlow for EEG Tasks (Signals) Classification. (\emph{Undergraduate graduation project})}
{Implemented 20+ popular algorithms including DNN, CNN, RNN-based, GCN with hands-on tutorials.}
{Finished writing \emph{three papers} based on this project as shown in my \emph{Publications}.}
{Comprehensive codes for EEG signals processing and classification \emph{research}, and got 100+ GitHub stars.}

\hspace*{\fill} 

\noindent \textbf{Shipwreck Sonar Image Segmentation based on Entropy Method} \href{https://github.com/SuperBruceJia/Sonar-Image-Segmentation-through-Entropy-Method}{[Code]} \hfill Jun - Sep 2018 
\workitemstwo
{Pre-processed sonar images to enhance the contract between the hull and reverberation area, which consists of discrete cosine filtering (DCT)$\rightarrow$edge detection (Roberts Operator)$\rightarrow$threshold segmentation via a one-dimensional histogram to locate the ship$\rightarrow$morphological expansion by tapered concentric rings through Matlab.}
{The proposed method improved segmentation accuracy (86\%+) compared with that without the pre-processed stage (no more than 80\%) on dozens of sonar images.}

\hspace*{\fill} 

%\noindent \textbf{Third China Data Mining Competition, Butterfly Recognition} \href{https://ccdm2018.sdufe.edu.cn/info/1012/1212.htm}{[Website]}  \href{https://github.com/SuperBruceJia/YOLO-V2-Object-Detection-Implementation}{[Code]} \hfill Oct 2017 - May 2018 
%\workitemsthree
%{Successfully implemented YOLO-V2 and Faster-RCNN object detection algorithms under Windows \& Ubuntu OS using the official butterfly dataset.}
%{Used affine transformation, noise addition, contrast enhancement, rotation, symmetry changing, and other methods to extend the dataset.}
%{YOLO-V2 Results (Team A106): 71.42\% Averaged IoU, and unsatisfactory classification accuracy.}
%
%\hspace*{\fill} 

\section{Awards}
\noindent 2019 Interdisciplinary Contest In Modeling (United States) \textbf{\emph{Honorable Mention}} \hfill Apr 2019\\
2018 Mathematical Contest In Modeling (Jilin, China) \textbf{\emph{First Prize}} \hfill Aug 2018\\
2018 Interdisciplinary Contest In Modeling (United States) \textbf{\emph{Successful Participant}} \hfill Apr 2018\\
2018 NEEPU Outstanding Student Leader \hfill Oct 2018\\
Innovation Scholarship of NEEPU \textbf{\emph{Winner}} \hfill 2018/2019\\
Academic Excellence Scholarship of NEEPU \textbf{\emph{Third Prize}} \hfill 2017/2018/2019/2020\\
%Jilin City International Marathon, Half Marathon, Placed \emph{148 / 5000} \hfill Jun 2017\\
%\emph{Student Member} of IEEE, ACM, and CCF \hfill 2019/2020/2021
2015 National High School Math League (Shanxi, China) \textbf{\emph{Second Prize}} \hfill Sep 2015
%The 32$^{nd}$ Chinese Physics Olympiad \textbf{\emph{Third Prize}} \hfill Oct 2015

\hspace*{\fill}

\section{Professional Skills}
\noindent \textbf{Languages}: Proficient in Python and Matlab; Familiar with C, R, HTML and JavaScript \\
\textbf{Libraries}: TensorFlow, PyTorch \\
\textbf{Other frequently-used tools}: Git, Vim, Markdown, Shell, \LaTeX, Docker and K8s \\
\textbf{English Language}: Fluent in English, CET-6 581, CET-4 577, Duolingo 110

\end{document} ​
