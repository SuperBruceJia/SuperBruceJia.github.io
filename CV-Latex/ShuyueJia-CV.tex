\documentclass{my_cv}

\begin{document}

\hspace*{\fill}

\name{Shuyue Jia} 

\hspace*{\fill}

\longcontact
{\href{mailto:shuyuej@ieee.org}{\nolinkurl{shuyuej@ieee.org}}}
{(+852) 5460-4494}
{\href{https://github.com/SuperBruceJia}{GitHub}}
{\href{https://resume.github.io/?SuperBruceJia}{GitHub Résumé}}
{\href{http://shuyuej.com/blog}{Blog}}
{\href{https://shuyuej.com/}{Personal Webpage}}

\hspace*{\fill} 

\section{Education}
\datedsubsection{City University of Hong Kong, Hong Kong, China}{}{May 2021 - Present}
\workitemsthree
{M.Phil., Computer Science Major, GPA: 3.50/4.0}
{Supervisor: Dr. Shiqi Wang, Research area: Image Quality Assessment and Perceptual Optimization}
{Selected coursework: Big Data Algorithms and Techniques (A), Machine Learning Algorithms and Applications (B)}

\datedsubsection{Northeast Electric Power University, Jilin, China}{}{Sep 2016 - Jun 2020}
\workitemstwo
{B.Eng., Intelligence Science and Technology Major, GPA: 80.26/100}
{Supervisor: Prof. Yimin Hou and Dr. Jinglei Lv, Research area: EEG Signals Processing based on DL Methods}

\datedsubsection{University of California, Irvine, CA, USA}{}{Jul - Sep 2017}
\workitemstwo
{Summer School, Computer Science, GPA: 4.0/4.0}
{Selected coursework: Computer Systems and Architecture (A+), University Writing and Communication (Pass)}

\hspace*{\fill}

\section{Research}

\workitemsone
{No-reference Image Quality Assessment via Non-local Modeling. \\
	\textbf{Shuyue Jia}, Baoliang Chen, Dingquan Li, Shiqi Wang. \\
	To be submitted.
}

\hspace*{\fill}

\workitemsone
{Deep Feature Mining via Attention-based BiLSTM-GCN for Human Motor Imagery Recognition. \href{https://www.frontiersin.org/articles/10.3389/fbioe.2021.706229/abstract}{[Paper]}\href{https://github.com/SuperBruceJia/EEG-DL}{[Code]}\\
	Yimin Hou, \textbf{Shuyue Jia (Corresponding Author)}, Xiangmin Lun, Shu Zhang, Jinglei Lv. \\
	\emph{Frontiers in Bioengineering and Biotechnology}, 2021. (Published)
}

\hspace*{\fill}

\workitemsone
{A Novel Approach of Decoding EEG Four-Class Motor Imagery Tasks via Scout ESI and CNN. \href{https://iopscience.iop.org/article/10.1088/1741-2552/ab4af6/meta}{[Paper]} \href{https://github.com/SuperBruceJia/EEG-Motor-Imagery-Classification-CNNs-TensorFlow}{[Code]}\\
	Yimin Hou, Lu Zhou, \textbf{Shuyue Jia}, and Xiangmin Lun. \\
	\emph{Journal of Neural Engineering}, 2020; 17(1):016048. (Published)
}

\hspace*{\fill}

\workitemsone
{GCNs-Net: A Graph Convolutional Neural Network Approach for Decoding Time-resolved EEG Motor Imagery Signals. \href{https://arxiv.org/abs/2006.08924}{[Paper]} \href{https://shuyuej.com/files/GCNs-Net.pdf}{[Spectral GNN Presentation]} \href{https://shuyuej.com/files/Dynamic-GCN-Survey.pdf}{[Dynamic GNN Presentation]} \href{https://github.com/SuperBruceJia/EEG-DL}{[Code]}\\
	Yimin Hou, \textbf{Shuyue Jia (Corresponding Author)}, Xiangmin Lun, Shu Zhang, Jinglei Lv. \\
	\emph{arXiv preprint arXiv:2006.08924}, 2022 (Rejected by IEEE TNSRE; To be submitted).
}

\hspace*{\fill} 

\workitemsone
{Improving Performance: a Collaborative Strategy for the Multi-data Fusion of Electronic Nose and Hyperspectral to Track the Quality Difference of Rice. \href{https://www.sciencedirect.com/science/article/abs/pii/S0925400521001143}{[Paper]} \\
	Yan Shi, Hangcheng Yuan, Chenao Xiong, \textbf{Shuyue Jia}, Jingjing Liu, and Hong Men.\\
	\emph{Sensors \& Actuators: B. Chemical}, 2021; 129546. (Published)
}
	
\hspace*{\fill}

\workitemsone
{Origin Traceability of Rice based on an Electronic Nose Coupled with a Feature Reduction Strategy. \href{https://iopscience.iop.org/article/10.1088/1361-6501/abb9e7/meta}{[Paper]} \\
	Yan Shi, Xiaofei Jia, Hangcheng Yuan, \textbf{Shuyue Jia}, Jingjing Liu, and Hong Men.\\
	\emph{Measurement Science and Technology}, 2020; 32(2):025107. (Published)
}
	
\hspace*{\fill}

\workitemsone
{Attention-based Graph ResNet for Motor Intent Detection from Raw EEG signals. \href{https://arxiv.org/abs/2007.13484}{[Paper]}\href{https://github.com/SuperBruceJia/EEG-DL}{[Code]}\\
	\textbf{Shuyue Jia (Corresponding Author)}, Yimin Hou, Yan Shi, and Yang Li.\\
	\emph{arXiv preprint arXiv:2007.13484}, 2022. (Rejected by MICCAI 2020; To be submitted)
}
	
\hspace*{\fill}

\section{Experience}

\datedsubsection{City University of Hong Kong}{Part-time Research Assistant}{Sep 2021 - April 2022}
\workitemstwo
{Investigated the topic of Deep Learning models compression and lightweight, and the deployment on mobile devices.}
{Funded by my mentor (Dr. Shiqi Wang) to cover tuition fees and living expenses in HK.}

\hspace*{\fill}

\datedsubsection{Tencent Video, Beijing}{Recommender System Intern}{Oct - Dec 2020}
\workitemstwo
{Assisted with the unified architecture regarding the \emph{Rerank} Module for Tencent Video Recommendation System.}
{Conducted research on the Dynamic Graph Convolutional Neural Networks (DGCN) \href{https://shuyuej.com/files/Dynamic-GCN-Survey.pdf}{Survey} and learned Reinforcement Learning models.}

\hspace*{\fill} 

\datedsubsection{Philips Research, Shanghai}{NLP Research Intern}{Jul - Oct 2020}
\workitemsthree
{\href{https://github.com/SuperBruceJia/Medical-Concept-Mapping}{Medical Concept Mapping}: three levels $\rightarrow$ BPE and FMM \& BMM Algorithms for Sub-words (Syntax-level), Word Vector Cosine Similarity (Semantics-level), and Knowledge Graph (Pragmatics-level).}
{\href{https://github.com/SuperBruceJia/MedicalNER}{Medical NER}: compared the performances of different models $\rightarrow$ CRF++, Character-level BiLSTM + CRF, Character-level BiLSTM + Word-level BiLSTM / CNNs + CRF, and deployed the models using Flask and Docker as web apps. Codes are available \href{https://github.com/SuperBruceJia/pytorch-flask-deploy-webapp}{here} and the Docker Images are available on the \href{https://hub.docker.com/u/shuyuej}{Docker Hub}.}
{\href{https://github.com/SuperBruceJia/dynamic-web-crawlering-python}{Dynamic Webs Crawling}: learned and crawled 620,000 words from NSTL using Python parallel package threading and other tricks to prevent Anti-reptile. (Mentor: Dr. Shuang Zhou)}

\hspace*{\fill} 

\datedsubsection{Tsinghua University, Beijing}{NLP Summer Intern}{Jun - Aug 2019}
\workitemsthree
{Natural Language Processing (NLP) Interned at State Key Laboratory of Intelligent Technology and Systems (Prof. Xiaoyan Zhu Team), Tsinghua University, China.}
{I was in a team that was responsible for building a salesman training system, which was a piece of insurance dialogue systems. During intern, I led the effort to create a \href{https://github.com/SuperBruceJia/Chinese-Chat-Title-NER-BERT-BiLSTM-CRF}{Chinese Chat Title Named Entity Recognition (NER)} via the BERT-BiLSTM-CRF model, and then matched the formal name with the recognized title through rules. \\NER Dataset: 30,676 samples, 96.73\% accuracy on 550 samples.}
{I also assisted in testing the sales training review system, and integrated salesman’s dialogue according to different difficulty levels, in verifying the reliability of the system.}

\hspace*{\fill} 

\section{Selected Projects}

\noindent \textbf{EEG-DL: A Deep Learning library for EEG Tasks (Signals) Classification} \href{https://github.com/SuperBruceJia/EEG-DL}{[Code]} \hfill May 2020 
\workitemsfour
{EEG-DL is a Deep Learning (DL) library written by TensorFlow for EEG Tasks (Signals) Classification.}
{Implemented 20+ popular algorithms including DNN, CNN, RNN-based, GCN with hands-on tutorials.}
{Finished writing \emph{three papers} based on this project as shown in my \emph{Publications}.}
{Comprehensive codes for EEG signals processing and classification \emph{research}, and got 300+ GitHub stars.}

\hspace*{\fill} 

\noindent \textbf{Shipwreck Sonar Image Segmentation based on Entropy Method} \href{https://github.com/SuperBruceJia/Sonar-Image-Segmentation-through-Entropy-Method}{[Code]} \hfill Jun - Sep 2018 
\workitemstwo
{Pre-processed sonar images to enhance the contract between the hull and reverberation area, which consists of discrete cosine filtering (DCT)$\rightarrow$edge detection (Roberts Operator)$\rightarrow$threshold segmentation via a one-dimensional histogram to locate the ship$\rightarrow$morphological expansion by tapered concentric rings through Matlab.}
{The proposed method improved segmentation accuracy (86\%+) compared with that without the pre-processed stage (no more than 80\%) on dozens of sonar images.}

\hspace*{\fill}

\section{Awards}

\noindent 2021 Standard Chartered Hong Kong Marathon, Half Marathon, Placed \emph{318 / 6000} (01:38:14) \hfill Oct 2021 \\
2019 Interdisciplinary Contest In Modeling, USA \href{https://shuyuej.com/files/ICM-2019.pdf}{[Thesis]}  \textbf{\emph{Honorable Mention}} \hfill Apr 2019 \\
2018 Mathematical Contest In Modeling, Jilin, China \href{https://shuyuej.com/files/MCM-2018.pdf}{[Thesis (in Mandarin)]}  \textbf{\emph{First Prize}} \hfill Aug 2018 \\
2018 Interdisciplinary Contest In Modeling, USA \href{https://shuyuej.com/files/ICM-2018.pdf}{[Thesis]}  \textbf{\emph{Successful Participant}} \hfill Apr 2018 \\
2018 NEEPU Outstanding Student Leader \hfill Oct 2018 \\
Innovation Scholarship of NEEPU  \textbf{\emph{Winner}} \hfill 2018/2019 \\
Academic Excellence Scholarship of NEEPU  \textbf{\emph{Third Prize}} \hfill 2017/2018/2019/2020 \\
Jilin City International Marathon, Half Marathon, Placed \emph{148 / 5000} (01:47:36) \hfill Jun 2017 \\
3000-meter Steeplechase (The $45^{th}$ NEEPU Games), The $7^{th}$ Place \hfill May 2017 \\
The 32$^{nd}$ Chinese Physics Olympiad, China  \textbf{\emph{Third Prize}} \hfill Oct 2015 \\
2015 National High School Math League, China  \textbf{\emph{Second Prize}} \hfill Sep 2015

\hspace*{\fill}

\section{Professional Skills}

\noindent \textbf{Languages}: Proficient in Python and Matlab; Familiar with C++/C/Embedded C, R, and JavaScript \\
\textbf{Libraries}: TensorFlow, PyTorch \\
\textbf{Other frequently-used tools}: Git, Vim, Markdown, Shell, \LaTeX, Docker, K8s \\
\textbf{English Language}: Fluent in English, CET-6 581, CET-4 577, Duolingo 110

\end{document}